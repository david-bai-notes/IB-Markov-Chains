\section{Another Application and Outlook}
How can one obtain information about a Markov chain from the observations of it?
Suppose $(X_i)_{i=0,\ldots,n}$ is given as observations.
For any candidate $\tilde{P}=(\tilde{p}_{ij})$, we define
$$l(\tilde{P})=\log(\tilde{p}_{X_0X_1}\tilde{p}_{X_1X_2}\cdots\tilde{p}_{X_{n-1}X_n})=\sum_{i,j\in I}N_{ij}(n)\tilde{p}_{ij}$$
where $N_{ij}(n)=\sum_{m=0}^{n-1}1_{X_m=i,X_{m+1}=j}$.
The maximum likelihood estimator $\hat{P}=\hat{P}(n)$ is the maximiser of $l$.
\begin{lemma}
    We have $\hat{p}_{ij}(n)=N_{ij}(n)/V_i(n)$ where like in the last section $V_i(n)=\sum_{k=0}^{n-1}1_{X_k=i}$.
\end{lemma}
\begin{proof}
    Use your favourite optimisation method, like Lagrange multipliers.
\end{proof}
\begin{theorem}
    If $P$ is positive recurrent, then
    $$\mathbb P[\hat{p}_{ij}(n)\to p_{ij}\text{ as }n\to\infty]=1$$
\end{theorem}
\begin{proof}
    Consider $N_{ij}=\sum_{m=1}^{V_i}Y_m$ where $Y_m$ is the indicator of the $m^{th}$ transition being from $i$ to $j$ and $V_i=V_i(\infty)$.
    By the Markov property, $Y_i$ are i.i.d. with mean $p_{ij}$ and independent from $V_i$.
    As $P$ is positive recurrent, $\mathbb P[V_i(n)\to\infty,n\to\infty]=1$, so the strong law of large numbers gives
    $$\mathbb P\left[ \hat{p}_{ij}(n)=\frac{1}{V_i(n)}\sum_{i=1}^{V_i(n)}Y_i\to p_{ij}\text{ as }n\to\infty \right]=1$$
    as desired.
\end{proof}
The course is basically finished here.
We might as well take a outlook on the subject.\\
For an aperiodic and irreducible finite-state Markov chain, we have seen that $\mathbb P[X_n=i]\to\pi_i$ as $n\to\infty$.
Thus, conversely, to sample from a given distribution $\pi$ (on say $N$ states), one may try to find a Markov chain with $\pi$ as its invariant distribution and then run it for a long time.
This is known as a Markov Chain Monte Carlo (MCMC) studies in the 50s by Metropolis \& Ulam.
There are different ways to find such a Markov chain.
The most famous one is Metropolis' algorithm originated from an article by Metropolis, Rosenbluth, Rosenbluth, Teller \& Teller (1953).
We will not explain it here -- you can look it up elsewhere.\\
There is another question of theoretical and practical relevance, which is the rate of this convergence.
For example, we might want to know more about
$$\min\{n:\sum_i|\mathbb P[X_n=i]-\pi_1|<\epsilon\}$$
This depends very much on the particular structure of the Markov chain and is expected to be highly nontrivial.
In fact, this is a subject of current research interest.\\
There are other perspectives of Markov chains we can study, for example we can look at random walks on groups, or consider continuous time and mixing time of Markov chains.