\section{Recurrence and Transience of Random Walks}
\begin{example}[Simple random walk on $\mathbb Z$]
    Consider the Markov chain with $I=\mathbb Z$ with $p_{i,i+1}=p,p_{i,i-1}=q=1-p$ where $p\in (0,1)$.
    We shall try to compute $p_{00}^{(N)}$.
    Now if $N$ is odd, then obviously $p_{00}^{(N)}=0$ by a parity argument.
    If $N=2n,n\in\mathbb Z$,
    $$p_{00}^{(2n)}=\binom{2n}{n}p^nq^n=\frac{(2n)!}{(n!)^2}p^nq^n$$.
    Stirling's formula gives $n!\sim\sqrt{2\pi n}e^{-n}n^n$ for large $n$ which gives
    $$p_{00}^{(2n)}\sim\frac{\sqrt{4\pi n}}{2\pi n}\frac{(2n)^{2n}}{n^{2n}}(pq)^n=\frac{C}{\sqrt{n}}(4pq)^n$$
    where $C$ is a constant we do not really care about at the moment.
    In the symmetric case $p=q=1/2$, we have $p_{00}^{(2n)}\sim C/\sqrt{n}\ge C/(2\sqrt{n})$ for $n\ge n_0$ for some $n_0$.
    Hence
    $$\sum_{n=0}^\infty p_{00}^{(n)}\ge \sum_{n=n_0}p_{00}^{(2n)}\ge \frac{C}{2}\sum_{n=n_0}^\infty \frac{1}{\sqrt{n}}=\infty$$
    so the random walk is recurrent.
    Otherwise $p\neq q$, in which case necessarily $4pq<(2p+2q)/2=1$ by AM-GM, so $p_{00}^{(2n)}\le r^n$ for $n\ge n_0$, so $\sum_np_{00}^{(n)}$ converges, therefore this random walk is transient.
    Hence symmetric random walks on $\mathbb Z$ is recurrent while asymmetric ones are transient.
\end{example}
\begin{example}[Simple symmetric random walk on $\mathbb Z^2$]
    Take $I=\mathbb Z^2$ and
    $$p_{ij}=\begin{cases}
        1/4\text{, if $|i-j|=1$}\\
        0\text{, otherwise}
    \end{cases}$$
    We will use a nice trick, which sadly doesn't work in high dimensions in general.
    Suppose $X_0=0$ and write $X_n^{\pm}$ as the orthogonal projections of $X_n$ onto the diagonal lines $y=\pm x$.
    Now $X_n^\pm$ are independent symmetric random walks in $2^{-1/2}\mathbb Z$.
    Also $X_0=0$ iff $X_0^{\pm}=0$.
    So by Stirling's approximation.
    $$p_{00}^{(2n)}=\left( \binom{2n}{n}\left(\frac{1}{2}\right)^{2n} \right)^2\sim\frac{C}{n}$$
    for some constant $C$.
    But then by summing them up we know this random walk is also recurrent.
\end{example}
Some na\"ive kids will then conjecture that any symmetric random walks on $\mathbb Z^n$ is recurrent.
Well, they are called na\"ive for a reason.
\begin{example}[Simple symmetric random walk on $\mathbb Z^3$]
    Consider $I=\mathbb Z^3$ and
    $$p_{ij}=\begin{cases}
        1/6\text{, if $|i-j|=1$}\\
        0\text{, otherwise}
    \end{cases}$$
    Again $p_{00}^{(N)}=0$ for odd $N$.
    All walks from $0$ to $0$ must take the same number of steps in directions $\pm e_i$, so
    \begin{align*}
        p_{00}^{(2n)}&=\sum_{i,j,k\ge 0,i+j+k=n}\binom{2n}{i,i,j,j,k,k}\left( \frac{1}{6} \right)^{2n}\\
        &=\binom{2n}{n}\left(\frac{1}{2} \right)^{2n}\sum_{i,j,k\ge 0,i+j+k=n}\binom{n}{i,j,k}^2\left( \frac{1}{3} \right)^{2n}
    \end{align*}
    Now if $n=3m$, it can be easily seen that $\binom{n}{i,j,k}$ is at most $\binom{n}{m,m,m}$.
    Also, we have
    $$\sum_{i,j,k\ge 0,i+j+k=n}\binom{n}{i,j,k}\left( \frac{1}{3} \right)^n=1$$
    by the multinomial theorem.
    These facts then implies that if $n=3m$,
    $$p_{00}^{(2n)}\le\binom{2n}{n}\left( \frac{1}{2} \right)^{2n}\binom{3m}{m,m,m}\left( \frac{1}{3} \right)^{3m}\sim Cn^{-3/2}$$
    In addition, we know that $p_{00}^{(2n)}\ge (1/6)^2p_{00}^{(2n-2)}$, so we can modify our $C$ by a factor to obtain $p_{00}^{(2n)}\le Cn^{-3/2}$ for large enough $n$, which shows
    $$\sum_{n}p_{00}^{(2n)}\le C\sum_{n}n^{-3/2}<\infty$$
    Hence this random walk is transcient.
\end{example}