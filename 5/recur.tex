\section{Recurrence and Transience}
\begin{definition}
    Let $(X_n)$ be a Markov chain.
    A state $i\in I$ is recurrent if
    $$\mathbb P_i[X_n=i\text{ infinitely often}]=1$$
    It is transient if
    $$\mathbb P_i[X_n=i\text{ infinitely often}]=0$$
\end{definition}
Recall the first passage time to $j$ is $T_j=\inf\{n\ge 1:X_n=j\}$.
\begin{theorem}\label{recur_trans}
    If $\mathbb P_i[T_i<\infty]=1$, then $i$ is recurrent and
    $$\sum_{n=0}^\infty p_{ii}^{(n)}=\infty$$
    Otherwise, $i$ is transient and
    $$\sum_{n=0}^\infty p_{ii}^{(n)}<\infty$$
\end{theorem}
\begin{corollary}
    Any state is either recurrent or transient, but not both.
\end{corollary}
\begin{proof}
    Immediate from the theorem.
\end{proof}
To prove the theorem, we need to do some ground work.
Inductively, we define the $r^{th}$ passage time to $j$ to be $T_j^{(0)}=0$ and
$$T_j^{(r+1)}=\inf\{n\ge T_j^{(r)}+1:X_n=j\}$$
Easily $T_j^{(1)}=T_j$.
The length of the $r^{th}$ excursion is defined by
$$S_i^{(r)}=\begin{cases}
    T_i^{(r)}-T_i^{(r-1)}\text{, if $T_i^{(r-1)}<\infty$}\\
    0\text{, otherwise}
\end{cases}$$
\begin{lemma}\label{tisi}
    For $r=2,3,\ldots$, conditional on $T_i^{(r-1)}<\infty$, the length of the $r^{th}$ excursion $S_i^{(r)}$ is independent of $\{X_m:m<T_i^{(r-1)}\}$ and
    $$\mathbb P[S_i^{(r)}=n|T_i^{(r-1)}<\infty]=\mathbb P_i[T_i=n]$$
\end{lemma}
\begin{proof}
    Quite obvious, but let's write it.
    Conditional on $T_i^{(r-1)}<\infty$, by Theorem \ref{strong_markov}, we have
    $$(X_{T_i^{(r-1)}+n})_{n\ge 0}\sim\operatorname{Markov}(\delta_i,P)$$
    and is independent of $X_0,\ldots,X_{T_i^{(r-1)}}$.
    Now we can rewrite
    $$S_i^{(r)}=\inf\{n\ge 1:X_{T_i^{(r-1)}+n}=i\}$$
    which makes it the first passage time of $(X_{T_i^{(r-1)}+n})_{n\ge 0}$.
    We are essentially done.
\end{proof}
Now let $V_i$ denote the number of visits to $i$, that is
$$V_i=\sum_{n=0}^\infty 1_{X_n=i}\implies \mathbb E_i[V_i]=\mathbb E_i\left[ \sum_{n=0}^\infty 1_{X_n=i} \right]=\sum_{n=0}^\infty\mathbb P_i[X_n=i]=\sum_{n=0}^\infty p_{ii}^{(n)}$$
Let $f_i$ be the return probability to $i$, that is $f_i=\mathbb P_i[T_i<\infty]$.
\begin{lemma}\label{vifi}
    For $r=0,1,2,\ldots$, we have $\mathbb P_i[V_i>r]=f_i^r$.
\end{lemma}
\begin{proof}
    Note that $\{V_i>r\}=\{T_i^{(r)}<\infty\}$ if $X_0=i$.
    Also note that $\mathbb P_i[V_i>0]=1$.
    If $\mathbb P_i[V_i>r]=f_i^r$ for some $r$, then by Lemma \ref{tisi},
    \begin{align*}
        \mathbb P_i[V_i>r+1]&=\mathbb P_i[T_i^{(r+1)}<\infty]\\
        &=\mathbb P_i[T_i^{(r)}<\infty,S_i^{(r+1)}<\infty]\\
        &=\mathbb P_i[T_i^{(r)}<\infty]\mathbb P[S_i^{(r+1)}<\infty|T_i^{(r)}<\infty]\\
        &=f_i^rf_i=f_i^{r+1}
    \end{align*}
    We conclude this by induction.
\end{proof}
\begin{proof}[Proof of Theorem \ref{recur_trans}]
    If $\mathbb P_i[T_i<\infty]=1$, then by Lemma \ref{vifi},
    $$\mathbb P_i[V_i=\infty]=\lim_{r\to\infty}\mathbb P_i[V_i>r]=1$$
    So $i$ is recurrent and $\sum_{n}p_{ii}^{(n)}=\mathbb E_iV_i=\infty$.\\
    Otherwise, $\mathbb P_i[T_i<\infty]<1$, then
    $$\sum_{n=0}^\infty p_{ii}^{(n)}=\mathbb E_iV_i=\sum_{r=0}^\infty\mathbb P_i[V_i>r]=\sum_{i=0}^\infty f_i^r=\frac{1}{1-f_i}<\infty$$
    In particular $\mathbb P_i[V_i=\infty]=0$, hence $i$ is transient.
\end{proof}
\begin{theorem}
    Recurrence and transience are class properties.
    That is, in any communicating class, either all states there is recurrent or all are transient.
\end{theorem}
\begin{proof}
    Let $i,j\in C$ and assume that $i$ is transient.
    Since $i,j$ communicate, there is some $n,m$ such that $p_{ij}^{(n)}>0,p_{ji}^{(m)}>0$.
    Then, for all $r\ge 0$ we have $p_{ii}^{(n+m+r)}\ge p_{ij}^{(n)}p_{jj}^{(r)}p_{ji}^{(m)}$.
    Assuming the transience of $i$, then
    $$\sum_{r=0}^\infty p_{jj}^{(r)}\le\frac{1}{p_{ij}^{(n)}p_{ji}^{(m)}}\sum_{r=0}^\infty p_{ii}^{(n+m+r)}<\infty$$
    So $j$ is transient.
    This is sufficient to imply the theorem.
\end{proof}
\begin{definition}
    A recurrent class is a communicating class which contains a recurrent state.
\end{definition}
By the preceding theorem, any state in a recurrent class is recurrent.
\begin{theorem}\label{recur_closed}
    Every recurrent class is closed.
\end{theorem}
\begin{proof}
    Let $C$ be a class that is not closed, then there is $i\in C,j\notin C$ and $m\ge 1$ such that $\mathbb P_i[X_m=j]>0$.
    Since $C$ is a communicating class and $j\notin C$,
    $$\mathbb P_i[\{X_m=j\}\cap\{X_n=i\text{ infinitely often}\}]=0$$
    So
    \begin{align*}
        \mathbb P_i[X_n=i\text{ infinitely often}]&=\sum_{j\in I}\mathbb P_i[\{X_m=j\}\cap\{X_n=i\text{ infinitely often}\}]\\
        &<\sum_{j\in I}\mathbb P_i[X_m=j]\\
        &=1
    \end{align*}
    So $i$ is transient, therefore not recurrent, as desired.
\end{proof}
\begin{theorem}\label{fin_closed_recur}
    Every finite closed class is recurrent.
\end{theorem}
Note that infinite closed classes can be transient.
\begin{proof}
    Start with any finite closed class $C$ and suppose $X_0\in C$.
    Then for some $i\in C$,
    \begin{align*}
        0&<\mathbb P[X_n=i\text{ infinitely often}]\\
        &=\mathbb P[X_n=i\text{ for some $n$}]\mathbb P_i[X_n=i\text{ infinitely often}]
    \end{align*}
    by Theorem \ref{strong_markov}.
    Hence
    $$\mathbb P_i[X_n=i\text{ infinitelty often}]>0$$
    Hence $i$ is not transient, so $i$ is recurrent, therefore $C$ is recurrent.
\end{proof}
\begin{corollary}
    Finite classes are recurrent if and only if they are closed.
\end{corollary}
\begin{proof}
    Combine Theorem \ref{recur_closed} and Theorem \ref{fin_closed_recur}.
\end{proof}
\begin{theorem}
    Suppose a Markov chain $P$ is irreducible and recurrent.
    Then for all $j\in I$, $\mathbb P[T_j<\infty]=1$.
\end{theorem}
\begin{proof}
    It suffices to show that $\mathbb P_i[T_j<\infty]=1$ for all $i\in I$ since that would imply
    $$\mathbb P[T_j<\infty]=\sum_{i\in I}\mathbb P[X_0=i]\mathbb P_i[T_j<\infty]=1$$
    As $P$ is irreducible, there is $m$ such that $p_{ji}^{(m)}>0$.
    Since $P$ is recurrent,
    $$1=\mathbb P_j[X_n=j\text{ infinitely often}]\le \mathbb P_j[X_n=j\text{ for some $n\ge m+1$}]$$
    So it implies the right hand side is $1$.
    Expanding it gives
    \begin{align*}
        1&=\sum_{k\in I}\mathbb P[X_n=j\text{ for some $n\ge m+1$}|X_m=k]\mathbb P_j[X_m=k]\\
        &=\sum_{k\in I}\mathbb P_k[X_n=j\text{ for some $n\ge 1$}]p_{jk}^{(m)}\\
        &=\sum_{k\in I}\mathbb P_k[T_j<\infty]p_{jk}^{(m)}
    \end{align*}
    Consequently $\mathbb P_k[T_j<\infty]=1$ for any $k\in I$.
    This completes the proof.
\end{proof}